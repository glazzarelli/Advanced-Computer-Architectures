%! Author = lazza
%! Date = 09/05/2022

\section{Performance and Cost}\label{sec:performance-and-cost}
When we say that one computer is faster than another what do we mean?
It depends on what is important.

Two metrics:
\begin{description}
    \item[Computer system user] minimize elapsed time for program execution:\\
    \textbf{latency:} (response time) execution time = time\_end - time\_start
    \item[Computer center manager] maximize completion rate \(=\frac{\#jobs}{\sec}\)\\
    \textbf{throughput:} total amount of work done in a given time
\end{description}
Throughput \(= \frac{1}{latency}\) only if there is not overlap between instructions, otherwise throughput >
response time.

\subsection{Measures}\label{subsec:measures}
\begin{gather*}
    \text{"X is k\% faster than Y"} = \frac{\text{execution time (y)}}{\text{execution time (x)}} = 1 + \frac{n}{100}\\
    \text{\textbf{Performance (x)}} = \frac{1}{\text{execution time(x)}}\\
    \Rightarrow \text{"X is k\% faster than Y"} = \frac{\text{performance (x)}}{\text{performance (y)}} = 1 +
\frac{n}{100}\\
\end{gather*}

\[\text{\textbf{Speed up (x,y)}} = \frac{\text{performance (x)}}{\text{performance (y)}}\]

\subsection{Amdahl's Law}\label{subsec:amdahl's-law}
Suppose that an enhancement E accelerates a fraction F of the task by a factor S (speedup enhanced), and the remainder
of the task in unaffected.

\begin{figure}[H]
    \centering
    \includegraphics[scale = 0.3]{images/amdahls-law}
    \caption{In green the (F) fraction enhanced.}
    \label{fig:amdalhs-law}
\end{figure}

\begin{gather*}
    ExTime_{new} = ExTime_{old} \times \left( \left( 1 - F_{en} \right) +
\frac{F_{en}}{S_{en}} \right)\\
    \mathbf{S_{overall}} = \frac{ExTime_{old}}{ExTime_{new}} = \frac{1}{\left( 1-F_{en} \right) +
\frac{F_{en}}{S_{en}}}\\
    \mathbf{Speedup_{overall(max)}} = \frac{1}{\left( 1 - F_{en} \right)}\\
\end{gather*}

This means that if an enhancement is only usable for a fraction
of a task we can’t speed up the task by more
than the reciprocal of 1 minus the fraction.
An upgrade is worth it if \(costs < speedup_{overall}\).

\textbf{Note:} in case of threads:
\[\mathbf{Speedup_{overall}} = \frac{1}{s + \frac{p}{N}}\\\]
s = serial part = 1 - Fraction enhanced\\
p = 1 - s = parallelized part = Fraction enhanced\\
N = number of processors or threads = Speedup


\subsection{CPU time and CPI}\label{subsec:cpu-time-and-cpi}
For computer architects:
\begin{center}
    Response time = CPU time + I/O wait
\end{center}

The \textbf{CPU time} does not include I/O wait time and corresponds to the time spent running the program.

\begin{center}
    \begin{align*}
        \text{CPU time (P)} &= \frac{\text{cc needed to execute P}}{\text{clock frequency}}\\
                     &= \text{cc needed to execute P} \times \text{cc time}
    \end{align*}
\end{center}

\begin{center}
    \textbf{CPI} \(=\frac{cycles}{instruction}\)
\end{center}

\begin{center}
    \textbf{IC} \(=\frac{instructions}{program}\)
\end{center}

\begin{center}
    \textbf{Cycle Time} \(=\frac{time}{seconds}\)
\end{center}

\begin{center}
    \textbf{CPU time} \(= IC \times CPI \times Cycle\_Time = \frac{seconds}{program}\)
\end{center}

\subsection{MIPS and MFLOPS}\label{subsec:mips-and-mflops}
MIPS has its shortcomings: it counts every instruction as if they were all equal, CPI varies with different
instructions.
\begin{align*}
    \text{\textbf{MIPS}} &= \text{millions of instructions per second}\\
                         &= \frac{\text{number of instructions}}{\text{execution time} \times 10^6}\\
                         &= \frac{\text{clock frequency}}{\text{CPI}\times 10^6}
\end{align*}

MFLOPS takes care of the problems related to MIPS and considers only the number of floating point operations, which we
assume
independent from compiler and ISA.

\begin{center}
    \textbf{MFLOPS} \(= \frac{\text{Floating point operations in program}}{\text{CPU time} \times 10^6}\)
\end{center}
