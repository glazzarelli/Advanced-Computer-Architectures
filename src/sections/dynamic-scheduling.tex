%! Author = lazza
%! Date = 03/05/2022

\subsection{Dynamic Scheduling}\label{subsec:dynamic-scheduling}

Scoreboard algo for dynamic scheduling

when is it safe to Issue an instruction slide 13
-
-
-
-

scoreboard manages commits to registers

RAW detected in ID stage, ID stage divided in two:
- instruction decode
- read registers

Scoreboard:
RAW ID stalls
WAR WB stalls
WAW not issued (stall the issue) - register renaming not used

techniques:
- register renaming

Scoreboard control structure:
1. instruction status, tells witch istr is being executed
2. functinal status, state of the functional unit
3. register result status, indicated which functional unit will write each register.

Qj and Qk tells who is going to provide the data in terms of functional unit (if not available yet)
Rj and Rk tells if the data is ready or not, prevents WAR

ADDD, DIVD, MULTD what D stands for?

in-order issue
out-of-order read
out-of-order write

-----

Tomasulo algo

load and store treated as integer FU

Common Data Bus - serialized access to write back
CDB provides values before they are saved into registers (somewhat similar to path forwarding)

FP floating point?

Reservation Station RS components
-
-
-
\ldots

STAGES
- issue
-
-

V value
Q pointer

LD: 1 cc to integer operation (offset + base address) and 1 cc to access the memory = 2 cc for Execution Completion + 1 cc to commit in the CDB

R(F4) = F4 value, register f4 Renamed

in-order issue
out-of-order execution
out-of-order write

try to do exercise on the slides

in case of concurrent writes choose the one that belongs to the critical path.

Scoreboard vs Tomasulo:
- structural hazards in scoreboard
- lack of forwarding in scoreboard

